\documentclass[12pt]{article}
\usepackage[utf8]{inputenc}
\usepackage[T1]{fontenc}
\usepackage{graphicx}
\usepackage{xcolor}

\include{defs}

\usepackage{lipsum}

%%%%%%%%%%%%%%%
% Title Page
\title{Abschlussbericht Autopilot}
\author{Hannes Marien \newline Torsten Noack \newline Hans Meyer \newline FlyHinotori | https://github.com/FlyHinotori}
\date{\today}
%%%%%%%%%%%%%%%

\begin{document}
\maketitle

\tableofcontents
\clearpage

\section{Problemstellung}

Das betreiben eines Flugzeugcharters bringt hohe Anforderungen für Ressourcenverwaltung, Abrechnung und Durchführung mit sich. Um den vielfältigen Aufgaben möglichst einfach gerecht zu werden und dem Kunden höchste Qualität zu bieten ist der Einsatz von Software für diesen Zweck sinnvoll. Mit Autopilot haben wir eine Software erstellt die diesen Anforderungen gerecht werden soll.
\newline

Besonders wenn es um die tägliche Anwendung geht, sollte dabei der Benutzer so umstandslos wie möglich durch den Prozess geführt werden. Deshalb haben wir die Benutzeroberfläche mit besonderem Schwerpunkt auf die Bedienbarkeit gestaltet. Ein optimales Ergebnis wurde erzielt, wenn auch ein Kunde die Erstellung eines Auftrags ohne Hilfe durchführen kann. Dadurch wäre gewährleistet das die Bedienung intuitiv ist und die Interna der Verwaltung der Daten keine Rolle für den Benutzer spielt.
\newline

Einfache Mechanismen der Auftragserstellung werden automatisch ausgeführt, der Benutzer muss sich darum nicht erst aufwändig selbst kümmern. Der Erstellung einer Rechnung für einen Auftrag oder die nötigen Schreiben für eine Stornierung/Änderung eines Auftrags werden automatisch erstellt und dem Benutzer zur Verfügung gestellt.
\newline

Da in professionellen Umgebungen mit Windows Betriebssystemen zu rechnen ist, wurde die Anwendung in C# für das .NET 4.5 erstellt. Damit ist maximale Kompatibilität in üblichen Umgebungen gewährleistet. Zudem gibt es eine Reihe von Technologien um die Anwednung bei Bedarf auch einfach und schnell für Linux, MacOS oder für den mobilen Bereich zur Verfügung zu stellen. Um den Datenschutz und die Kontrolle über die eigenen Daten sicherzustellen haben wir von einer Cloudlösung abgesehen. Alle Benutzer arbeiten auf ihrer eigenen Datei-basierten Datenbank.

\subsection{Pictures used}

\noindent
Cover picture filename (in titlepage): \texttt{cover}\\
Logo filename (in foot): \texttt{logo}

\subsection{Boxes}

\begin{verbatim}
\fullboxbegin
Content
\fullboxend
\end{verbatim}

\begin{verbatim}
\leftboxbegin
Content
\leftboxend
\end{verbatim}

\begin{verbatim}
\rightboxbegin
Content
\rightboxend
\end{verbatim}

\begin{verbatim}
\frameboxbegin{Frame Title}
Content
\frameboxend
\end{verbatim}

\newpage

\section{First section}
\lipsum[1]

\fullboxbegin
\lipsum[1]
\fullboxend

\lipsum[1]

\subsection{First subsection}
\lipsum[1]

\leftboxbegin
Lorem ipsum dolor sit amet, consectetuer adipiscing elit. Ut purus elit, vestibulum ut, placerat ac, adipiscing vitae, felis. Curabitur dictum gravida mauris. Nam arcu libero, nonummy eget, consectetuer id, vulputate a, magna. Donec vehicula augue eu neque. 
\leftboxend

\lipsum[1-2]

\rightboxbegin
\begin{itemize}
 \item Lorem ipsum
 \item Lorem ipsum
\end{itemize}
\rightboxend

\lipsum[1]

\subsubsection{First subsubsection}

\lipsum[1]

\begin{figure}[!h]
\centering
\includegraphics[width=0.5\textwidth]{sky.jpg}
\caption{The sky is the limit.}
\end{figure}

\section*{Unnumbered section}
\lipsum[1]

\begin{figure}[!h]
\centering
\includegraphics[width=0.5\textwidth]{sky.jpg}
\caption*{The sky is the limit.}
\end{figure}

\section{Second section}

\lipsum[1]
\begin{table}[!h]
\centering
\caption{Sample table.}
\begin{tabular}{cccc}
\toprule
Value 1 & Value 2 & Value 3 & Value 4\\
\midrule
 odd     & odd   & odd & 1.00 \\
 even    & even  & even& 1.00 \\
 odd     & odd   & odd & 1.00 \\
 even    & even  & even& 1.00 \\
\bottomrule
\end{tabular}
\end{table}

\lipsum[1]

\frameboxbegin{Sample frame}
\lipsum[1]
\frameboxend

\end{document}          
