\documentclass[12pt]{article}
\usepackage[utf8]{inputenc}
\usepackage[T1]{fontenc}
\usepackage{graphicx}
\usepackage{xcolor}

\include{defs}

\usepackage{lipsum}

%%%%%%%%%%%%%%%
% Title Page
\title{Abschlussbericht Autopilot}
\author{Hannes Marien \newline Torsten Noack \newline Hans Meyer \newline FlyHinotori | https://github.com/FlyHinotori}
\date{\today}
%%%%%%%%%%%%%%%

\begin{document}
\maketitle

\tableofcontents
\clearpage

\section{Problemstellung}

Das betreiben eines Flugzeugcharters bringt hohe Anforderungen für Ressourcenverwaltung, Abrechnung und Durchführung mit sich. Um den vielfältigen Aufgaben möglichst einfach gerecht zu werden und dem Kunden höchste Qualität zu bieten ist der Einsatz von Software für diesen Zweck sinnvoll. Mit Autopilot haben wir eine Software erstellt die diesen Anforderungen gerecht werden soll.
\newline

Besonders wenn es um die tägliche Anwendung geht, sollte dabei der Benutzer so umstandslos wie möglich durch den Prozess geführt werden. Deshalb haben wir die Benutzeroberfläche mit besonderem Schwerpunkt auf die Bedienbarkeit gestaltet. Ein optimales Ergebnis wurde erzielt, wenn auch ein Kunde die Erstellung eines Auftrags ohne Hilfe durchführen kann. Dadurch wäre gewährleistet das die Bedienung intuitiv ist und die Interna der Verwaltung der Daten keine Rolle für den Benutzer spielt.
\newline

Einfache Mechanismen der Auftragserstellung werden automatisch ausgeführt, der Benutzer muss sich darum nicht erst aufwändig selbst kümmern. Der Erstellung einer Rechnung für einen Auftrag oder die nötigen Schreiben für eine Stornierung/Änderung eines Auftrags werden automatisch erstellt und dem Benutzer zur Verfügung gestellt.
\newline

Da in professionellen Umgebungen mit Windows Betriebssystemen zu rechnen ist, wurde die Anwendung in C# für das .NET 4.5 erstellt. Damit ist maximale Kompatibilität in üblichen Umgebungen gewährleistet. Zudem gibt es eine Reihe von Technologien um die Anwednung bei Bedarf auch einfach und schnell für Linux, MacOS oder für den mobilen Bereich zur Verfügung zu stellen. Um den Datenschutz und die Kontrolle über die eigenen Daten sicherzustellen haben wir von einer Cloudlösung abgesehen. Alle Benutzer arbeiten auf ihrer eigenen Datei-basierten Datenbank.

\newpage
\section{Anforderungen}

"Autopilot" sollte eine Anwendung für die Planung, Durchführung und Abrechnung von Charterflügen sein. Die Erstellung von Flügen, dem Verwalten von entsprechenden Ressourcen und die automatische Abrechnung sind also die Kernfunktionen des Programms. Zusätzlich sollten Kunden und deren Zahlungen erfasst und verwaltet werden. Bei Zahlungsausfall war ein entsprechendes automatisches Mahnwesen gefordert. Kunden waren in verschiedene Kategorien einzuteilen die zusätzliche Eigenschaften mit sich bringen. Da fast alle Interaktion mit Kunden entsprechende Dokumente nach sich zieht, hatten wir dafür zu sorgen das auch diese ohne Zutun des Benutzers erstellt und von der Anwendung verwaltet wird.
\newline

Nach langer Besprechnung zu den entsprechenden Anforderungsbereichen haben wir zusätzlich implizite Anforderungen indentifiziert:
\begin{itemize}
  \item Benutzerfreundlichkeit
  \item Datenschutz
  \item offener Entwicklungsprozess
  \item Verfügbarkeit in Standardumgebungen
\end{itemize}
\newline

Diese impliziten Anforderungen bringen eine Reihe von Konsequenzen mit sich die nicht sofort offensichtlich sind und deshalb im Folgenden kurz erläutert werden sollen.

\subsection{Benutzerfreundlichkeit}

"Autopilot" ist eine Anwendung zum Verwalten von Daten. Dazu eignen sich Computer grundsätzlich besser als Menschen, da sie nicht müde werden und keine Flüchtigkeitsfehler machen. Die Daten zur Verarbeitung werden dennoch von Menschen eingegeben, wahrscheinlich mehrmals am Tag und wahrscheinlich nicht immer in ruhiger Atmosphäre.
\newline

Entsprechend war unser erstes Ziel dem Benutzer die Navigation im Programm und dessen Bedienung so leicht wie möglich zu machen. Wir haben dazu folgende Schwerpunkte erarbeitet:
\begin{itemize}
  \item Einfache Navigation in der Oberfläche
  \item Für den aktuellen Vorgang unwichtige Informationen ausblenden
  \item Komplexe Schritte in Teilschritte aufteilen
\end{itemize}
\newline

Die meisten Benutzer sind sehr optisch veranlagt, weshalb wir uns in der Navigation darauf geeinigt haben Icons zu verwenden statt Text. Das erhöht die Wiedererkennung und beschleunigt die Suche nach dem richtigen Button, da nicht jede Aufschrift erst gelesen werden muss. Zudem variieren die Positionen der einzelnen Buttons nicht. Da sich nicht alle wesentlichen Elemente in der grafischen Oberfläche durch gängige Symbole ersetzen lassen, sollte in den entsprechenden Bereichen lieber eine gewöhnliche Schlatfläche mit sprechender Bezeichnung verwendet werden.
\newline

In der Oberfläche sollte es möglich sein die Stammdaten zu bearbeiten. Dazu soll jeder wesentliche Aspekt des Programms der in der Datenbank abgelegt wird auch aus der grafischen Oberfläche verfügbar sein. Diese Anforderung zog auch Anforderungen an die Datenbank nach sich, da dem Nutzer die Interna der gespeicherten Daten nicht offensichtlich werden sollten. Eine solche unnötige Komplexität für den Benutzer wollten wir ausblenden.
\newline

Die Funktion die über die Zeit am häufigsten genutzt werden dürfte, ist die Erstellung eines neuen Auftrags. Dieser Schritt umfasst die Erfassung des Kunden, die Auswahl der Route, die Zuteilung des benötigten Personals, die Feststellung der Verfügbarkeit aller Ressourcen, die Einordnung des Kunden in die entsprechende Kundenkategorie und die entsprechenden Abrechnungsmodalitäten des Auftrags. Da eine derartige Komplexität für einen Benutzer nicht leicht, wenn überhaupt, zu handhaben ist, sollte die grafische Benutzeroberfläche nicht auch noch zusätzliche Schwierigkeiten mit sich bringen. Einfache Lesbarkeit und fokussierte Schritte sollten dem Benutzer helfen den Auftrag zu erstellen. Entsprechend sollten die einzelnen Schritte des Auftrags nur Informationen und Eingabemasken anbieten die den aktuellen Schritt betreffen. Zwischen den einzelnen Schritten kann bis zum Ende der Erfassung des Auftrags beliebig hin und her gesprungen werden. Informationen die der Datenbank noch nicht bekannt sind werden automatisch erstellt, so das die Aufnahme neuer Kunden in die Datenbank der Auftragserstellung nicht hinderlich im Weg steht. Im Ergebnis soll der Prozess fließend von der Hand gehen und auch für Ungeübte nachvollziehbar sein.

\subsection{Datenschutz}

Kundendaten, Geschäftsgeheimnisse oder Liquiditätsinformationen werden heutezutage häufiger Ziel von Hackerangriffen. Auch die beste Cloudinfrastruktur ist davor nicht geschützt und so haben wir uns überlegt wie wir diese Informationen so gut wie möglich schützen können.
\newline

Zunächst sollte dazu die Kontrolle über Speicherort und Backup der Daten in Ihrer Hand liegen und nicht in unserer. Zudem sollte ein einfaches Format den Austausch der Datenbank erleichtern.
\newline

Ihnen ist damit die Möglichkeit gegeben Ihre Daten unter den Mitarbeitern zu teilen wenn sie wollen oder auf einem Einzelplatz zu arbeiten. Im Zeitalter der Cloudanbieter finden sich eine ganze Reihe von Diensten die Dateien automatisch auf angeschlossenen Geräten synchronisieren und die gleichzeitig als Backup zum Einsatz kommen können. Als Beispiel sei hier der Anbieter SpiderOak genannt, der ähnlich wie Dropbox arbeitet, jedoch auf Seite der Endbenutzer verschlüsselt und deshalb kein Wissen über die Inhalte der gespeicherten Daten erlangen kann.
\newline 

Durch diese einfachen Mechanismen obliegt es Ihren Präferenzen den Datenschutz sicherzustellen und die Aufwändige Implementierung einer zusätzlichen Infrastruktur kann entfallen. Mit der Variante eines Cloudanbieters wie SpiderOak wäre zusätzlich die Frage nach einem Backup und einem Mehrbenutzersystem geklärt. Wir beraten Sie gerne bei zusätzlichen Fragen in diesem Bereich.

\subsection{Entwicklungsprozess}

Für ein schnelles Feedback und Ihre aktive Teilnahme an der Projektgestaltung haben wir Ihnen jederzeit offenen Zugriff auf alle unsere Quellen und Ergebnisse geben wollen. Auf diese Weise konnten Sie jederzeit korrigierend eingreifen wenn die Entwicklung eine falsche Richtung einschlug oder sie mit einem Zwischenstand nicht einverstanden waren. Wir möchten uns an dieser Stelle erneut für die Rege Teilnahme bedanken und haben uns gefreut das die von uns vorgeschlagenen Möglichkeiten von Ihnen zugelassen wurden. 
\newline

Alle unsere Informationen sollten an zentraler Stelle zusammenlaufen. In unserem Fall war das der Slack Chat in dem für alle Teilnehmer offensichtlich wurde wer gerade an welchem Teil des Programms arbeitet. Hier sollten auch entsprechende Frage geklärt und die Ergebnisse diskutiert werden. Durch die Integration mit unserem Code-Host GitHub wurden auch die Nachrichten aus dem Issue-Tracker und der Codereview für alle sichtbar.
\newline

Für das kontinuierliche Testen des Programms wollten wir einen Continuous Integration Dienst verwenden. Dieser hatte jede Änderungen unabhängig vom jeweiligen Entwickler zu bauen und festzustellen ob die Änderungen immernoch ein fertiges Programm ergeben. Flüchtigkeitsfehler oder Probleme durch Inkomptibilitäten wurden damit sofort Offensichtlich. Durch die Entscheidung für AppVeyor hatten wir zusätzlich die Möglichkeit für eine ganze Reihe von verschiedenen Windows Versionen und Architekturen entsprechende Testergebnisse zu beziehen.

\subsection{Anwendungsumgebung}

\newpage
\section{Projektumgebung}
\subsection{Aufgabenteilung}

\newpage
\section{Durchführung}
\subsection{Planung}
\subsection{Durchführungsprobleme und Lösungen}

\newpage
\section{Dokumentation}

\newpage
\section{Projektzukunft}

\end{document}          
