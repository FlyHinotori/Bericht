\documentclass[12pt]{article}
\usepackage[utf8]{inputenc}
\usepackage[T1]{fontenc}
\usepackage{graphicx}
\usepackage{xcolor}

\include{defs}

\usepackage{lipsum}

%%%%%%%%%%%%%%%
% Title Page
\title{Abschlussbericht Autopilot}
\author{Hannes Marien \newline Torsten Noack \newline Hans Meyer \newline FlyHinotori | https://github.com/FlyHinotori}
\date{\today}
%%%%%%%%%%%%%%%

\begin{document}
\maketitle

\tableofcontents
\clearpage

\section{Problemstellung}

Das betreiben eines Flugzeugcharters bringt hohe Anforderungen für Ressourcenverwaltung, Abrechnung und Durchführung mit sich. Um den vielfältigen Aufgaben möglichst einfach gerecht zu werden und dem Kunden höchste Qualität zu bieten ist der Einsatz von Software für diesen Zweck sinnvoll. Mit Autopilot haben wir eine Software erstellt die diesen Anforderungen gerecht werden soll.
\newline

Besonders wenn es um die tägliche Anwendung geht, sollte dabei der Benutzer so umstandslos wie möglich durch den Prozess geführt werden. Deshalb haben wir die Benutzeroberfläche mit besonderem Schwerpunkt auf die Bedienbarkeit gestaltet. Ein optimales Ergebnis wurde erzielt, wenn auch ein Kunde die Erstellung eines Auftrags ohne Hilfe durchführen kann. Dadurch wäre gewährleistet das die Bedienung intuitiv ist und die Interna der Verwaltung der Daten keine Rolle für den Benutzer spielt.
\newline

Einfache Mechanismen der Auftragserstellung werden automatisch ausgeführt, der Benutzer muss sich darum nicht erst aufwändig selbst kümmern. Der Erstellung einer Rechnung für einen Auftrag oder die nötigen Schreiben für eine Stornierung/Änderung eines Auftrags werden automatisch erstellt und dem Benutzer zur Verfügung gestellt.
\newline

Da in professionellen Umgebungen mit Windows Betriebssystemen zu rechnen ist, wurde die Anwendung in C# für das .NET 4.5 erstellt. Damit ist maximale Kompatibilität in üblichen Umgebungen gewährleistet. Zudem gibt es eine Reihe von Technologien um die Anwednung bei Bedarf auch einfach und schnell für Linux, MacOS oder für den mobilen Bereich zur Verfügung zu stellen. Um den Datenschutz und die Kontrolle über die eigenen Daten sicherzustellen haben wir von einer Cloudlösung abgesehen. Alle Benutzer arbeiten auf ihrer eigenen Datei-basierten Datenbank.

\newpage
\section{Anforderungen}

"Autopilot" sollte eine Anwendung für die Planung, Durchführung und Abrechnungn von Charterflügen sein. Die Erstellung von Flügen, dem Verwalten von entsprechenden Ressourcen und die automatische Abrechnung sind also die Kernfunktionen des Programms. Zusätzlich sollten Kunden und deren Zahlungen erfasst und verwaltet werden. Bei Zahlungsausfall war ein entsprechendes automatisches Mahnwesen gefordert. Kunden waren in verschiedene Kategorien einzuteilen die zusätzliche Eigenschaften mit sich bringen. Da fast alle Interaktion mit Kunden entsprechende Dokumente nach sich zieht, hatten wir dafür zu sorgen das auch diese ohne Zutun des Benutzers erstellt und von der Anwendung verwaltet wird.
\newline

Nach langer Besprechnung zu den entsprechenden Anforderungsbereichen haben wir zusätzlich implizite Anforderungen indentifiziert:
\begin{itemize}
  \item Benutzerfreundlichkeit
  \item Datenschutz
  \item offener Entwicklungsprozess
  \item Verfügbarkeit in Standardumgebungen
\end{itemize}
\newline

Diese impliziten Anforderungen bringen eine Reihe von Konsequenzen mit sich die nicht sofort offensichtlich sind und deshalb im Folgenden kurz erläutert werden sollen.

\subsection{Benutzerfreundlichkeit}

"Autopilot" ist eine Anwendung zum Verwalten von Daten. Dazu eignen sich Computer grundsätzlich besser als Menschen, da sie nicht müde werden und keine Flüchtigkeitsfehler machen. Die Daten zur Verarbeitung werden dennoch von Menschen eingegeben, wahrscheinlich mehrmals am Tag und wahrscheinlich nicht immer in ruhiger Atmosphäre.
\newline

Entsprechend war unser erstes Ziel dem Benutzer die Navigation im Programm und dessen Bedienung so leicht wie möglich zu machen. 
\newline
\subsection{Datenschutz}
\subsection{Entwicklungsprozess}
\subsection{Anwendungsumgebung}

\newpage
\section{Projektumgebung}
\subsection{Aufgabenteilung}

\newpage
\section{Durchführung}
\subsection{Planung}
\subsection{Durchführungsprobleme und Lösungen}

\newpage
\section{Dokumentation}

\newpage
\section{Projektzukunft}

\end{document}          
