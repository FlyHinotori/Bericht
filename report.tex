\documentclass[12pt]{article}
\usepackage[utf8]{inputenc}
\usepackage[T1]{fontenc}
\usepackage{graphicx}
\usepackage{xcolor}

\include{defs}

\usepackage{lipsum}

%%%%%%%%%%%%%%%
% Title Page
\title{Abschlussbericht Autopilot}
\author{Hannes Marien \newline Torsten Noack \newline Hans Meyer \newline FlyHinotori | https://github.com/FlyHinotori}
\date{\today}
%%%%%%%%%%%%%%%

\begin{document}
\maketitle

\tableofcontents
\clearpage

\section{Problemstellung}

Das Betreiben eines Flugzeugcharters bringt hohe Anforderungen für Ressourcenverwaltung, Abrechnung und Durchführung mit sich. Um den vielfältigen Aufgaben möglichst einfach gerecht zu werden und dem Kunden höchste Qualität zu bieten ist der Einsatz von Software für diesen Zweck sinnvoll. Mit Autopilot haben wir eine Software erstellt die diesen Anforderungen gerecht werden soll.
\newline

Besonders wenn es um die tägliche Anwendung geht, sollte dabei der Benutzer so umstandslos wie möglich durch den Prozess geführt werden. Deshalb haben wir die Benutzeroberfläche mit besonderem Schwerpunkt auf die Bedienbarkeit gestaltet. Ein optimales Ergebnis wurde erzielt, wenn auch ein Kunde die Erstellung eines Auftrags ohne Hilfe durchführen kann. Dadurch wäre gewährleistet das die Bedienung intuitiv ist und die Interna der Verwaltung der Daten keine Rolle für den Benutzer spielt.
\newline

Einfache Mechanismen der Auftragserstellung werden automatisch ausgeführt, der Benutzer muss sich darum nicht erst aufwändig selbst kümmern. Der Erstellung einer Rechnung für einen Auftrag oder die nötigen Schreiben für eine Stornierung/Änderung eines Auftrags werden automatisch erstellt und dem Benutzer zur Verfügung gestellt.
\newline

Da in professionellen Umgebungen mit Windows Betriebssystemen zu rechnen ist, wurde die Anwendung in C\# für das .NET 4.5 erstellt. Damit ist maximale Kompatibilität in üblichen Umgebungen gewährleistet. Zudem gibt es eine Reihe von Technologien um die Anwendung bei Bedarf auch einfach und schnell für Linux, MacOS oder für den mobilen Bereich zur Verfügung zu stellen. Um den Datenschutz und die Kontrolle über die eigenen Daten sicherzustellen haben wir von einer Cloudlösung abgesehen. Alle Benutzer arbeiten auf ihrer eigenen Datei-basierten Datenbank.

\newpage
\section{Anforderungen}

"Autopilot" sollte eine Anwendung für die Planung, Durchführung und Abrechnung von Charterflügen sein. Die Erstellung von Flügen, dem Verwalten von entsprechenden Ressourcen und die automatische Abrechnung sind also die Kernfunktionen des Programms. Zusätzlich sollten Kunden und deren Zahlungen erfasst und verwaltet werden. Bei Zahlungsausfall war ein entsprechendes automatisches Mahnwesen gefordert. Kunden waren in verschiedene Kategorien einzuteilen, die zusätzliche Eigenschaften mit sich bringen. Da fast alle Interaktion mit Kunden entsprechende Dokumente nach sich ziehen, hatten wir dafür zu sorgen, dass auch diese ohne Zutun des Benutzers erstellt und von der Anwendung verwaltet wird.
\newline

Nach langer Besprechung zu den entsprechenden Anforderungsbereichen haben wir zusätzlich implizite Anforderungen identifiziert:
\begin{itemize}
  \item Benutzerfreundlichkeit
  \item Datenschutz
  \item offener Entwicklungsprozess
  \item Verfügbarkeit in Standardumgebungen
\end{itemize}
\newline

Diese impliziten Anforderungen bringen eine Reihe von Konsequenzen mit sich, die nicht sofort offensichtlich sind und deshalb im Folgenden kurz erläutert werden sollen.

\subsection{Benutzerfreundlichkeit}

"Autopilot" ist eine Anwendung zum Verwalten von Daten. Dazu eignen sich Computer grundsätzlich besser als Menschen, da sie nicht müde werden und keine Flüchtigkeitsfehler machen. Die Daten zur Verarbeitung werden dennoch von Menschen eingegeben, wahrscheinlich mehrmals am Tag und wahrscheinlich nicht immer in ruhiger Atmosphäre.
\newline

Entsprechend war unser erstes Ziel dem Benutzer die Navigation im Programm und dessen Bedienung so leicht wie möglich zu machen. Wir haben dazu folgende Schwerpunkte erarbeitet:
\begin{itemize}
  \item Einfache Navigation in der Oberfläche
  \item Für den aktuellen Vorgang unwichtige Informationen ausblenden
  \item Komplexe Schritte in Teilschritte aufteilen
\end{itemize}
\newline

Die meisten Benutzer sind sehr optisch veranlagt, weshalb wir uns in der Navigation darauf geeinigt haben Icons zu verwenden statt Text. Das erhöht die Wiedererkennung und beschleunigt die Suche nach dem richtigen Button, da nicht jede Aufschrift erst gelesen werden muss. Zudem variieren die Positionen der einzelnen Buttons nicht. Da sich nicht alle wesentlichen Elemente in der grafischen Oberfläche durch gängige Symbole ersetzen lassen, sollte in den entsprechenden Bereichen lieber eine gewöhnliche Schaltfläche mit sprechender Bezeichnung verwendet werden.
\newline

In der Oberfläche sollte es möglich sein die Stammdaten zu bearbeiten. Dazu soll jeder wesentliche Aspekt des Programms der in der Datenbank abgelegt wird, auch aus der grafischen Oberfläche verfügbar sein. Diese Anforderung zog auch Anforderungen an die Datenbank nach sich, da dem Nutzer die Interna der gespeicherten Daten nicht offensichtlich werden sollten. Eine solche unnötige Komplexität für den Benutzer wollten wir ausblenden.
\newline

Die Funktion die über die Zeit am häufigsten genutzt werden dürfte, ist die Erstellung eines neuen Auftrags. Dieser Schritt umfasst die Erfassung des Kunden, die Auswahl der Route, die Zuteilung des benötigten Personals, die Feststellung der Verfügbarkeit aller Ressourcen, die Einordnung des Kunden in die entsprechende Kundenkategorie und die Abrechnungsmodalitäten des Auftrags. Da eine derartige Komplexität für einen Benutzer nicht leicht, wenn überhaupt, zu handhaben ist, sollte die grafische Benutzeroberfläche nicht auch noch zusätzliche Schwierigkeiten mit sich bringen. Einfache Lesbarkeit und fokussierte Schritte sollten dem Benutzer helfen den Auftrag zu erstellen. Entsprechend sollten die einzelnen Schritte des Auftrags nur Informationen und Eingabemasken anbieten die den aktuellen Schritt betreffen. Zwischen den einzelnen Schritten kann bis zum Ende der Erfassung des Auftrags beliebig hin und her gesprungen werden. Informationen die der Datenbank noch nicht bekannt sind, werden automatisch erstellt, sodass die Aufnahme neuer Kunden in die Datenbank der Auftragserstellung nicht hinderlich im Weg steht. Im Ergebnis soll der Prozess fließend von der Hand gehen und auch für Ungeübte nachvollziehbar sein.

\subsection{Datenschutz}

Kundendaten, Geschäftsgeheimnisse oder Liquiditätsinformationen werden heutzutage häufiger Ziel von Hackerangriffen. Auch die beste Cloudinfrastruktur ist davor nicht geschützt und so haben wir uns überlegt wie wir diese Informationen so gut wie möglich schützen können.
\newline

Zunächst sollte dazu die Kontrolle über Speicherort und Backup der Daten in Ihrer Hand liegen und nicht in unserer. Zudem sollte ein einfaches Format den Austausch der Datenbank erleichtern.
\newline

Ihnen ist damit die Möglichkeit gegeben Ihre Daten unter den Mitarbeitern zu teilen wenn sie wollen oder auf einem Einzelplatz zu arbeiten. Im Zeitalter der Cloudanbieter finden sich eine ganze Reihe von Diensten die Dateien automatisch auf angeschlossenen Geräten synchronisieren und die gleichzeitig als Backup zum Einsatz kommen können. Als Beispiel sei hier der Anbieter SpiderOak genannt, der ähnlich wie Dropbox arbeitet, jedoch auf Seite der Endbenutzer verschlüsselt und deshalb kein Wissen über die Inhalte der gespeicherten Daten erlangen kann.
\newline 

Durch diese einfachen Mechanismen obliegt es Ihren Präferenzen den Datenschutz sicherzustellen und die aufwändige Implementierung einer zusätzlichen Infrastruktur kann entfallen. Mit der Variante eines Cloudanbieters wie SpiderOak wäre zusätzlich die Frage nach einem Backup und einem Mehrbenutzersystem geklärt. Wir beraten Sie gerne bei zusätzlichen Fragen in diesem Bereich.

\subsection{Entwicklungsprozess}

Für ein schnelles Feedback und Ihre aktive Teilnahme an der Projektgestaltung haben wir Ihnen jederzeit offenen Zugriff auf alle unsere Quellen und Ergebnisse geben wollen. Auf diese Weise konnten Sie jederzeit korrigierend eingreifen, wenn die Entwicklung eine falsche Richtung einschlug oder sie mit einem Zwischenstand nicht einverstanden waren. Wir möchten uns an dieser Stelle erneut für die rege Teilnahme bedanken und haben uns gefreut, dass die von uns vorgeschlagenen Möglichkeiten von Ihnen zugelassen wurden. 
\newline

Alle unsere Informationen sollten an zentraler Stelle zusammenlaufen. In unserem Fall war das der Slack Chat in dem für alle Teilnehmer offensichtlich wurde, wer gerade an welchem Teil des Programms arbeitet. Hier sollten auch entsprechende Fragen geklärt und die Ergebnisse diskutiert werden. Durch die Integration mit unserem Code-Host GitHub wurden auch die Nachrichten aus dem Issue-Tracker und der Codereview für alle sichtbar.
\newline

Für das kontinuierliche Testen des Programms wollten wir einen Continuous Integration Dienst verwenden. Dieser hatte jede Änderungen unabhängig vom jeweiligen Entwickler zu bauen und festzustellen ob die Änderungen immernoch ein fertiges Programm ergeben. Flüchtigkeitsfehler oder Probleme durch Inkompatibilitäten wurden damit sofort Offensichtlich. Durch die Entscheidung für AppVeyor hatten wir zusätzlich die Möglichkeit, für eine ganze Reihe von verschiedenen Windows Versionen und Architekturen entsprechende Testergebnisse zu beziehen.

\subsection{Anwendungsumgebung}

Eine Festlegung auf ein bestimmtes Betriebssystem am Anfang eines Projekts ist nicht immer notwendig, kann die Entwicklung aber wesentlich beschleunigen. Wir hatten in Absprache mit Ihnen am Anfang festgelegt, dass nur Windows-Umgebungen in Frage kommen. Wir wollten dabei möglichst viele verschiede Windows Varianten unterstützen, Zukünftige wie auch Veraltete. Außerdem sollte die grafische Oberfläche sich wenigstens generell an die gängigen Annahmen der Windows-Plattform halten.
\newline

Um in diesen Belangen möglichst wenig selbst programmieren zu müssen und uns auf die wesentlichen Inhalte konzentrieren zu können, haben wir uns für das .NET Framework 4.5 als Laufzeitumgebung geeinigt. Dieses ist auf allen aktuellen Windows Varianten lauffähig und wird auch von zukünftigen Windows Generationen noch unterstützt werden. Es ist hinreichend schnell und bietet für viele alltägliche Programmieraufgaben bereits gute Abstraktionen.

\newpage
\section{Projektumgebung}

Durch die Anforderungen einer transparenten Projektentwicklung haben wir uns für eine Entwickler-freundliche Umgebung entscheiden können. Zum Projekt gehören die folgenden Webseiten:

\begin{itemize}
  \item https://github.com/FlyHinotori - und die Repositories dieser GitHub Organistation, inklusive der dort verzeichneten Issues und PullRequests
  \item https://github.io/FlyHinotori/docs - Statische Webseite mit der Präsentation zum ersten Zwischenbericht
  \item https://github.io/FlyHinotori/docs2 - Statische Webseite mit der Präsentation zum zweiten Zwischenbericht
  \item https://ci.appveyor.com/project/HaMster21/autopilot - Mit den Ergebnissen der Continuous Integration
  \item https://hino.slack.com - Als Kommunikationszentrale
  \item http://benutzerdoku.readthedocs.org/en/latest/index.html - Für die Benutzerdokumentation
\end{itemize}

Zentraler Teil ist das GitHub Repository https://github.com/FlyHinotori/autopilot. Dort lässt sich der gesamte Verlauf der Entwicklung am Programm nachvollziehen. Einige Issues beschreiben dort auch die Anforderungen aus dem Plichtenheft, allerdings sind wir in der Kommunikation auf Slack schneller gewesen was die Anforderungen angeht. Umgekehrt sind einige Probleme als Issue im Repository festgehalten die nicht auf Slack sondern bei GitHub diskutiert wurden. Da alle Benutzer in Slack benachrichtigt werden ist aber nichts untergegangen.
\newline

Thematisch zusammenhängende Änderungen haben wir versucht in PullRequests abzubilden. Da PullRequests automatisch auch vom ContinuousIntegration-Dienst gebaut werden, sind sie der optimale Ort für CodeReviews und Diskussionen über das umgesetzte Feature gewesen.
\newline

Die Entwicklung auf den Arbeitsplätzen der Entwickler fand mit Visual Studio 2013 statt. Da wir alle auf unterschiedlich leistungsfähigen Computern mit verschiedener Ausstattung und verschiedenen Windows Versionen gearbeitet haben konnten wir während der Entwicklung bereits verschiedene Integrationstests für die spätere Umgebung durchführen.

\subsection{Aufgabenteilung}

Der Projektleiter für diesen Auftrag war Hannes Marien. Zusammen mit Hans Meyer und Torsten Noack bildete er auch das Entwicklerteam und leitete die Kommunikation mit Ihnen, dem Auftraggeber.
\newline

Aufgrund bestehender Präferenzen der Entwickler war es recht leicht Bereiche abzustecken um die sich die einzelnen Mitglieder gekümmert haben. Bei drei Entwicklern ergab das drei Verantwortlichkeiten: 

\begin{itemize}
  \item Datenbank und Persistenz - Torsten Noack
  \item Programmlogik und Anwendungsfälle - Hannes Marien
  \item Grafische Oberfläche und Abhängigkeitsmanagement - Hans Meyer
\end{itemize}

Da es große Überschneidungen der einzelnen Felder gab hat jeder von uns in jedem Feld gearbeitet, die Verantwortlichkeiten waren also eine Richtlinie um im Zweifel einen Ansprechpartner benennen zu können.
\newline

Die Durchführung hatte noch die Einrichtung der entsprechenden Dienste zur Folge. Slack und AppVeyor wurden im Projektverlauf von Hans Meyer eingerichtet und verwaltet. 
\newline

Dokumentation und Zwischenberichte sind in gemeinschaftlicher Arbeit entstanden. Auch hier haben die entsprechenden Verantwortlichen die Kerngedanken zu ihrem Verantwortungsbereich beigetragen und wir haben sie in gemeinschaftlicher Arbeit zu einem fertigen Ergebnis gebracht.
\newline

\newpage
\section{Durchführung}
\subsection{Planung}

Vor dem Projektstart haben wir Arbeitspakete definiert und unter den Entwicklern aufgeteilt. Seit dem zweiten Zwischenbericht haben wir uns auf offene Punkte konzentriert, Fehler im Programm minimiert und die Dokumentation entwickelt.
\newline 

Ihnen und uns ist während der Durchführung aufgefallen das die Planung andere Schwerpunkte gesetzt hat als in der Durchführung zu beobachten waren. So hat der Prozess der Erstellung der Software bis zum Projektabschluss angedauert und war nicht wie geplant am 10.6.2015 beendet. Da wir von Ihnen während dieser Zeit sehr viel wertvolles Feedback erhalten haben das wir berücksichtigen wollten, ist diese Abweichung aus unserer Sicht gerechtfertigt. Da wir viel mit Ihnen kommuniziert haben, gehen wir davon aus das Sie ebenso denken.
\newline

Mit der verlängerten Implementierungsphase ging auch die Verlängerung der Testphase und der Fehlerbereinigung einher. Da wir zum Abschluss ein Programm nach Ihren Wünschen übergeben wollten, sind diese Planungsbereiche auch bis zum Ende verfolgt und nicht zu geplanten Termin abgebrochen worden. Da wir auch während der Zwischenpräsentation noch kleine Fehler festgestellt hatten, konnten wir auch dadurch noch Anreize für Korrekturen finden.
\newline

Insgesamt haben wir das Projekt sehr dynamisch durchgeführt. Bezeichnend sind die Phasen intensiver Ausarbeitung der Anforderungen die sich mit Phasen produktiver Implementierung abgewechselt haben. Wir würden uns über eine Bewertung Ihrerseits freuen, damit wir für zukünftige Projekte an unseren Methoden und Hilfsmittel arbeiten können.

\subsection{Durchführungsprobleme und Lösungen}

Zu Beginn des Projekts haben wir begonnen eine Windows Universal-App zu programmieren. Dieser Projekttyp ist erst seit Oktober 2014 verfügbar und soll laut Microsoft Apps für alle Geräte ermöglichen. Die Unterscheidung zwischen Windows und Windows Phone würde dabei zum Beispiel entfallen, Universal-Apps wären auf beiden Arten von Betriebssystemen lauffähig. Dieser Projekttyp nutzt ebenfalls ein installiertes .NET Framework als Laufzeitumgebung und kann folglich auch mit allen Programmiersprachen umgehen die mit dem .NET zusammenarbeiten.
\newline

Da aber die Annahmen der Plattformen mobiler Geräte schon erheblich verschieden von den Desktop-Plattformen sind, gibt es einige Einschränkungen bei diesem Projekttyp. Zusätzlich sind auch Plattformen als Ziel für Windows Universal-Apps geplant die heute noch keine Rolle spielen. Virtual-Reality-Brillen beispielsweise. Auf derartigen Geräten und Smartphones ist ein Deployment einer Datenbank undenkbar. Entsprechend sind für Windows Universal-Apps einige unserer Kernkomponenten nicht verfügbar. Es fehlt zum Beispiel das Entity-Framework und die Windows Presentation Foundation classes.
\newline

Nach vielen Untersuchungen wie man diese Einschränkungen umgehen kann und der Auswahl einer Reihe von alternativen Komponenten sind wir zu dem Entschluss gekommen eine klassische WPF-Anwendung zu erstellen. Diese ist zwar nicht so flexibel, bietet aber alle Funktionalitäten die man für eine zügige Entwicklung brauchen kann. Damit konnten wir dann auch entsprechend schnell Ergebnisse liefern ohne das wir uns aufwändig mit der Dokumentation befassen mussten. Diese ist für Windows Universal-Apps derzeit noch sehr unstrukturiert, wenn überhaupt vorhanden. Unsere gewünschten Schnittstellenbibliotheken stehen für WPF-Anwendungen alle zur Verfügung und die Dokumentation dieser Plattform ist sehr gut.

\newpage
\section{Dokumentation}

Da wir Ihnen zu jeder Zeit den Stand der Dinge offensichtlich machen wollten, sind alle Dokumente und Präsentationen bei GitHub versioniert worden. GitHub bietet als Plattform die Möglichkeit automatisch statische Webseiten aus versioniertem Text zu erstellen. Die Einschränkungen dafür sind minimal. Auch die Möglichkeit mit WebHooks andere Prozesse anzustoßen hat uns hier sehr in die Hände gespielt.
\newline

Da wir in dieser Version keine Schnittstellen nach außen vorgesehen haben, bleibt das Projekt noch ohne Schnittstellenreferenz. Diese wäre aber im Bedarfsfall schnell aus dem Autopilot-Repository zu generieren.
\newline

Die Präsentationen zu den Zwischenständen finden Sie unter
\begin{itemize}
  \item https://github.com/FlyHinotori/docs und
  \item https://github.com/FlyHinotori/docs2.
\end{itemize}
Ein entsprechender Link zur statischen Webseite von GitHub findet sich in der Beschreibung des jeweiligen Repositories. Beide Präsentationen sind einfach HTML Webseiten mit einiger JavaScript-Logik. Ein einfaches Kopieren aller Dateien im Repository macht die Präsentation auch außerhalb von GitHub auf einem Computer ohne Internetzugang verfügbar. Wenn eine PDF Version gewünscht wird kann diese aus den Quellen generiert werden.
\newline

Die Benutzerdokumentation ist als Webseite verfügbar unter
\begin{itemize}
  \item http://benutzerdoku.readthedocs.org/en/latest/index.html.
\end{itemize}
Wann immer im Repository https://github.com/FlyHinotori/BenutzerDoku etwas geändert wird, löst ein WebHook eine neue Erstellung der Dokumentation auf readthedocs.org aus und der Benutzer kann sich den aktuellen Stand angucken. Es ist auch möglich die Dokumentation für eine bestimmte Version auszuwählen, da wir aber nur eine Version 1.0 haben kommt dieses Feature noch nicht zum Tragen. Mit readthedocs ist es zusätzlich möglich die Dokumentation als PDF herunterzuladen, was den Einsatz in Umgebungen ohne Internetzugang möglich macht.
\newline

Auch dieser Bericht ist als GitHub Repository angelegt und kann unter 
\begin{itemize}
  \item https://github.com/FlyHinotori/Bericht
\end{itemize}
eingesehen werden. Der Link in der README zeigt auf einen Dienst der daraus bei jedem Commit ein PDF Dokument generiert. Damit ist gewährleistet das die aktuelle Version für alle zur Verfügung steht. Die PDF-Datei kann ebenso heruntergeladen und offline verfügbar gemacht werden.

\newpage
\section{Projektzukunft}

Für den Fall das Sie eine weitere Zusammenarbeit wünschen haben wir während der Entwicklung einige Vorschläge gesammelt wie das Produkt "Autopilot" Ihre Arbeit in Zukunft noch stärker vereinfachen könnte.
\newline

Zunächst wäre die Einbindung einer Bankverbindung hilfreich. Über standardisierte Schnittstellen könnte man Zahlungseingänge und -ausgänge überwachen und in der Anwendung entsprechend darauf reagieren. Hierbei würde dann der manuelle Kontoabgleich entfallen der derzeit noch unumgänglich ist.
\newline

Weiterhin haben wir und die gewählten Hilfsmittel die Möglichkeit schnell und einfach auf eine zentrale Datenbank umzustellen. Diese würde die Mehrplatzarbeit erheblich vereinfachen. Zusätzlich wäre damit die Frage nach einem Backup innerhalb Ihres Netzwerks geklärt.
\newline

Wenn Sie einer Einbindung von Webdiensten oder Microservices zustimmen würden, könnte man auch Benachrichtigungen für die Beteiligten eines Charters anbieten. So wäre es etwa denkbar das Piloten und Kabinenpersonal per SMS oder WhatsApp über eine Buchung informiert werden und der Kunde neben einer Mail auch noch eine SMS als Flugerinnerung einen Tag vor dem Flug bekommt.
\newline

Wir sind gerne bereit unsere Vorschläge im Detail zu besprechen und eine neue Version mit Ihnen zu planen. Es würde uns freuen wenn sie weiterhin mit uns zusammenarbeiten wollen.

\end{document}          
